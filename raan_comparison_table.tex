\begin{table}[ht]
\centering
\caption{Comparação entre a precessão nodal teórica por $J_2$ e a medida numericamente a partir de $\Omega(t)$.}
\label{tab:raan_j2_comp}
\sisetup{table-number-alignment = center, round-mode = places, round-precision = 6}
\begin{tabular}{l S[table-format=5.0] S[table-format=1.5] S[table-format=3.3] S[table-format=1.6] S[table-format=1.6] S[table-format=+2.2] S[table-format=2.1]}
\toprule
Satélite & {$\bar{a}$\,[km]} & {$\bar{e}$} & {$\bar{i}$\,[deg]} & {$\dot{\Omega}_{\text{teo.}}$\,[deg/d]} & {$\dot{\Omega}_{\text{num.}}$\,[deg/d]} & {$\varepsilon$\,[\%]} & {Duração\,[d]} \\
\midrule
VH Up & 24241 & 0.70337 & 97.530 & 0.047787 & 0.096805 & 102.58 & 5.0 \\
V Only & 24235 & 0.70330 & 97.500 & 0.047623 & 0.048293 & 1.41 & 5.0 \\
VH Down & 24258 & 0.70358 & 97.471 & 0.047358 & 0.000771 & -98.37 & 5.0 \\
\bottomrule
\end{tabular}
\end{table}